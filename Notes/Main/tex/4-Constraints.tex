\chapter{Constraints}
Constraints place restrictions on the values (or set of values) that can be inserted into the database. They are \text{hard} constraints, meaning the DBMS will check and enforce them. They are metadata used to make explicit domain constraints, e.g. an age must be non-negative, and they maintain the integrity of the database, not just locally to the table, e.g. only record marks for students who are registered.

Examples of constraints are domain constraints, \\
e.g. \verb|INTEGER|, \verb|NOT NULL|, \verb|UNIQUE|.

\section{Keys}
\subsection{Primary Key}
By tagging an attribute as a primary key, it enforces the constraints \verb|UNIQUE| and \verb|NOT NULL|. It also signals to the DBMS that this is \textit{probably} an attribute that can be used to uniquely identify tuples.

A primary key constraint can be seen in the following example:
\begin{lstlisting}[language=SQL,tabsize=3]
CREATE TABLE Student (
	sid INTEGER,
	...
	PRIMARY KEY (sid)
)
\end{lstlisting}

And a relation with a primary key can be denoted as \texttt{Student(\underline{sid}, dob, login, course)}.

\subsubsection{Primary Key Example}
\begin{lstlisting}[language=SQL,tabsize=3]
CREATE TABLE Student (
	sid INTEGER,
	dob CHAR(10),
	login CHAR(20),
	course CHAR(10),
	PRIMARY KEY (sid)
)
\end{lstlisting}

which is the same as

\begin{lstlisting}[language=SQL, tabsize=3]
CREATE TABLE Student (
	sid INTEGER,
	dob CHAR(10),
	login CHAR(20),
	course CHAR(10),
	CONSTRAINT StudentsKey PRIMARY KEY (sid)
)
\end{lstlisting}

by naming the constraint, we are able to manipulate it later (we don't want rebuild the whole database!)

\subsection{Foreign Key}
A foreign key defines a link to another table. It usually specifies the primary key in that other table. Any row inserted into \textit{this} table must	satisfy the constraint that the foreign key in \textit{this} table must be matched by a key in the other table.

\subsubsection{Foreign Key Example}
\begin{lstlisting}[language=SQL,tabsize=3]
CREATE TABLE Student (
	sid INTEGER,
	dob CHAR(10),
	login CHAR(20),
	course CHAR(10),
	PRIMARY KEY (sid)
)

CREATE TABLE Marks (
	sid INTEGER,
	cid INTEGER,
	mark INTEGER,
	FOREIGN KEY (sid) REFERENCES Student (sid),
	FOREIGN KEY (cid) REFERENCES Course (cid)
)
\end{lstlisting}

This isn't just checked on insertion; it's guaranteed to \textit{always} be true.

\section{Deletion}
What happens if we want to delete a record? We can't have some fields referencing non-existent values in other tables. There are several operations to deal with deletion:

\begin{description}
	\item[ON DELETE RESTRICT] checks whether or not the deletion will put the system in an inconsistent state \textit{before} the deletion is performed.
	\item[ON DELETE NO ACTION] checks whether or not the deletion will put the system in an inconsistent state \textit{after} the deletion is performed.
	\item[ON DELETE CASCADE] propagates the deletion in one table throughout the other tables.
	\item[ON DELETE SET DEFAULT/NULL] instead of deleting the record, it gives it a default value, e.g. 999999.
\end{description}

So the integrity of the database is maintained by forbidding the deletion \\
(\verb|RESTRICT|, \verb|NO ACTION|), or by deleting rows that reference the deleted key (\verb|CASCADE|).

\subsubsection{Example of Deletion Control}
\begin{lstlisting}[language=SQL,tabsize=3]
CREATE TABLE Marks (
	sid INTEGER,
	cid INTEGER,
	mark INTEGER,
	
	FOREIGN KEY (sid) REFERENCES Student (sid),
		ON DELETE CASCADE,
		ON UPDATE CASCADE,
	FOREIGN KEY (cid) REFERENCES Course (cid)
)
\end{lstlisting}

The code indicates that 
\begin{itemize}
	\item changes to \verb|sid| in the Students table propagate through to the Marks table.
	\item changes to \verb|cid| in the Courses table are forbidden (the default action is \verb|NO ACTION|).
\end{itemize}

\section{Domain Constraints}
We can enforce additional constraints:
\begin{itemize}
	\item Local constraints
	\item Constraints over more than one table
\end{itemize}

For example:
\begin{lstlisting}[language=SQL,tabsize=3]
CREATE TABLE Marks (
	sid INTEGER,
	cid INTEGER,
	mark INTEGER,
	
	FOREIGN KEY (sid) REFERENCES Student (sid),
		ON DELETE CASCADE,
		ON UPDATE CASCADE,
	FOREIGN KEY (cid) REFERENCES Course (cid),
	CHECK (mark >= 0 and mark <= 100)
)
\end{lstlisting}

\section{Constraints: Summary}
\begin{itemize}
	\item Constraints express metadata about our domain---they are declarative, so no need to write code.
	\item They are guaranteed to be enforced (not just checks on insertion).
	\item When designing, capture all \textit{hard} domain constraints and express them in the database.
	\item Do not include constraints which are desirable but not inviolable, e.g. someone might not have a surname, so don't add the constraint \verb|NOT NULL| to the surname field.
\end{itemize}









